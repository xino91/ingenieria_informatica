\section{Introduccion a los modelos de redes neuronales}

Las redes neuronales son un modelo computacional basado en el funcionamiento de las neuronas. Consiste en un conjunto de unidades llamadas neuronas, conectadas entre sí para trasmitir señales. La información de entrada atraviesa la red neuronal produciendo unos valores de salida.
Este trabajo consiste en implementar el algoritmo de retropropagación para entrenar un perceptrón multicapa. Para ello se desarrollará un programa capaz de realizar este entrenamiento, con distintas posibilidades en cuanto a la parametrización del mismo. 

La arquitectura de la red (número de capas y número de
nodos en cada capa), junto con la tipología de los nodos
(tipos de funciones de activación a considerar), son
parámetros decisivos del algoritmo que hay que buscar por
prueba y error o por validación cruzada.

El sobre-entrenamiento en redes neuronales suele venir
provocado por dos causas:
Entrenamiento demasiado largo (condición de parada
inadecuada).
Redes demasiado complejas (muchas neuronas o muchas
capas).