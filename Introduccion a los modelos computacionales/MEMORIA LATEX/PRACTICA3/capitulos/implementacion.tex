\section{Implementación Algoritmo RBF}
\subsection{Arquitectura de los modelos}

\begin{itemize}
	\item Una capa de entrada con tantas neuronas como variables tenga la base de datos considerada.
	\item Una capa oculta con un número de neuronas a especificar por el usuario.  Siempre tendremos solo una capa oculta, con todas las neuronas de tipo RBF.
	\item Una capa de salida con tantas neuronas como variables de salida tenga la base de datos considerada.
	\begin{itemize}
		\item Si la base de datos es de regresión: Todas las neuronas serán de tipo lineal.
		\item Si la base de datos es de clasificación, todas las neuronas de la capa de salida serán de tipo Softmax.
	\end{itemize}
\end{itemize}

\subsection{Algoritmo Entrenamiento RBF}
\begin{algorithm}[H]
\caption{Entrenamiento RBF Offline}
\begin{algorithmic}[1]
	\STATE centroidesIniciales $\leftarrow$ aleatoriamente $n_1$ patrones 		     (regresion) o $n_1$ patrones de forma estratificada (clasificacion)
	\STATE centroides $\leftarrow$ K-medias (X,$n_1$,centroidesIniciales)
	\STATE $\sigma_j \leftarrow$ (medias de las distancias al resto de centroides)/2
	\STATE Construir la matriz $R_{(Nx(n1+1))}$, donde $R_{ij} = out_j^1(x_i)$ para $j\neq (n_1 +1)$, y $R_{ij} = 1$ para $j =(n_1 +1)$
	\IF {clasificacion}
		\STATE PesosSalida $\leftarrow$ AplicarRegresionLogistica(R,eta)
	\ELSE
		\STATE PesosSalida $\leftarrow$ CalcularPseudoInversa(R)
	\ENDIF
\end{algorithmic}
\end{algorithm}

\begin{itemize}
	\item[1.] Si es clasificación sacamos centroides de forma estratificada, y si es regresión escogemos aleatoriamente $n_1$ patrones
	\item[2.] Ajuste de centros mediante algoritmo de clustering más popular, el K-means, con tantos clusters como neuronas de la RBF.
	\item[3.] Para ajustar los radios de la RBF tomaremos la mitad de la distancia media al resto de centroides. $\sigma_j$ = $\dfrac{1}{2(n_1-1)} * \sum_{i\neq j} \lvert\lvert c_j-c_i \rvert\rvert$
	\item[4.] Construimos la matriz R, donde tenemos el valor de activación de cada neurona para cada patrón.
	\item[5.] Si es Clasificación
	\begin{itemize}
		\item[6.] Aplicamos regresión logística, utilizamos una función a la que le pasamos la matriz R. (puede incluir regularización, parámetro $\eta$).
	\end{itemize}
	\item[7.] Si es Regresión
	\begin{itemize}
		\item[8.] Una vez construida la matriz R, aplicamos la pseudo-inversa de Moore Penrose.
	\end{itemize}
\end{itemize}



